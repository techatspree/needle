%!TEX root = NeedleReference.tex
\chapter{Getting started}

\section{Sample Application}
\label{sec: Sample Application}
In this chapter, a very simple application is shown, which will be tested in the following chapters with Needle. 
The application contains a simple user management.   

\parskip 14pt
\parindent 0pt
There are two jpa entity classes - a User and a Profile, with OneToOne relationships between them. 
The following two listings shows the implementation of the data model.

\begin{lstlisting}[language={JAVA},caption=The user entity]
@Entity
public class User {

   @Id
   @GeneratedValue(strategy = GenerationType.AUTO)
   private Long id;

   @Column(unique = true, nullable = false)
   private String username;

   @Column(nullable = false)
   private String password;

   @OneToOne(optional = false, cascade = CascadeType.ALL)
   private Profile profile;

   public Long getId() {
      return id;
   }

   public String getUsername() {
      return username;
   }

   public void setUsername(String username) {
      this.username = username;
   }

   public String getPassword() {
      return password;
   }

   public void setPassword(String password) {
      this.password = password;
   }

   public Profile getProfile() {
      return profile;
   }

   public void setProfile(Profile profile) {
      this.profile = profile;
   }
}
\end{lstlisting}


\begin{lstlisting}[language={JAVA},caption=The profile entity]
@Entity
public class Profile {

   @Id
   @GeneratedValue(strategy = GenerationType.AUTO)
   private Long id;

   @Column(nullable = false)
   private String language;

   public String getLanguage() {
      return language;
   }

   public void setLanguage(String language) {
      this.language = language;
   }

   public Long getId() {
      return id;
   }
}
\end{lstlisting}


\parskip 14pt
\parindent 0pt
To access the user data there is a simple user dao component.

\begin{lstlisting}[language={JAVA},caption=The user dao component]
public class UserDao {

   @PersistenceContext
   private EntityManager entityManager;

   public User findBy(final String username, final String password) {
      CriteriaBuilder builder = entityManager.getCriteriaBuilder();
      CriteriaQuery<User> query = builder.createQuery(User.class);

      Root<User> user = query.from(User.class);

      query.where(
            builder.and(builder.equal(user.get(User_.username), username)),
            builder.equal(user.get(User_.password), password));

      return entityManager.createQuery(query).getSingleResult();
   }


   public List<User> findAll() {
      CriteriaBuilder builder = entityManager.getCriteriaBuilder();
      CriteriaQuery<User> query = builder.createQuery(User.class);

      return entityManager.createQuery(query).getResultList();
   }
}
\end{lstlisting}

\parskip 14pt
\parindent 0pt
To authenticate a user the application has an implementation of an authenticator component which use the user dao.

\begin{lstlisting}[language={JAVA},caption=The authenticator component]
public class Authenticator {

   @Inject
   private UserDao userDao;

   public boolean authenticate(final String username, final String password) {

      User user = userDao.findBy(username, password);

      return user != null ? true : false;

   }

}
\end{lstlisting}

\section{Using Needle with JUnit}
\label{sec: JUnit}

Needle provides JUnit Rules as extenstion for JUnit. Rules are basically wrappers around test methods. They can execute code before, after or instead of a test method.

The following example demonstrates a simple JUnit Needle test. In the Test case there are two Rules. The Database Rule provides access to the database via JPA and execute optional database operations. 
The NeedleRule prepare all fields annotated with @ObjectUnderTest and initialize the components under test. Needle supported injections are constructor injection, field injection and method injection.

\begin{lstlisting}[language={JAVA},caption=JUnit user dao test]

public class UserDaoTest {

   @Rule
   public DatabaseRule databaseRule = new DatabaseRule();

   @Rule
   public NeedleRule needleRule = new NeedleRule(databaseRule);

   @ObjectUnderTest
   private UserDao userDao;

   @Test
   public void testFindByUsername() throws Exception {
      final User user = new UserTestdataBuilder(
            databaseRule.getEntityManager()).buildAndSave();

      User findBy = userDao.findBy(user.getUsername(), user.getPassword());

      Assert.assertEquals(user.getId(), findBy.getId());
   }

   @Test
   public void testFindAll() throws Exception {
      new UserTestdataBuilder(databaseRule.getEntityManager()).buildAndSave();

      List<User> all = userDao.findAll();

      Assert.assertEquals(1, all.size());
   }
}
\end{lstlisting}

\section{Using Needle with TestNG}
\label{sec: TestNG}

Needle also supports TestNG. There are two test cases. The de.akquinet.jbosscc.needle.testng.AbstractNeedleTestcase base test class that prepare all fields annotated with @ObjectUnderTest and initialize the components under test. The de.akquinet.jbosscc.needle.testng.DatabaseTestcase can use as a special provider for the EntityManager injection or as a base test case for jpa tests.

\begin{lstlisting}[language={JAVA},caption=TestNG user dao test]
public class UserDaoTest extends AbstractNeedleTestcase {

   public UserDaoTest() {
      super(new DatabaseTestcase());
   }

   @ObjectUnderTest
   private UserDao userDao;

   @Test
   public void testFindByUsername() throws Exception {
      final User user = new UserTestdataBuilder(getEntityManager())
            .buildAndSave();

      User findBy = userDao.findBy(user.getUsername(), user.getPassword());

      Assert.assertEquals(user.getId(), findBy.getId());
   }

   @Test
   public void testFindAll() throws Exception {
      new UserTestdataBuilder(getEntityManager()).buildAndSave();

      List<User> all = userDao.findAll();

      Assert.assertEquals(1, all.size());
   }
}
\end{lstlisting}